
\section{Conclusion}
\label{sec:conclusion}

In this study, we have analyzed the Temporal Network dynamics of the Bitcoin transactions. We achieved that by using only a subset of the Blockchain to demonstrate a proof of concept. Then, we translate, the not so consistent, Bitcoin transaction data into a Temporal Network. From this Temporal Network, we characterize the dynamics of the agents by identifying 36 types of temporal motifs. This characterization of motifs proves to be very efficient to detect significant changes in the behavior of our Temporal Network. Using this metric, we tried to model the dynamics embedded in the Bitcoin Temporal Network subset. We used a state of the art model as our ground zero to generate networks based on sole a Distribution, capable enough of creating complex dynamics. From our several modeling trials, we generated a slightly improved model, compared to the ground zero model, by adding memory to the agents. However, we were not able to create a convincing model which could capture the whole dynamics of the Bitcoin transactions. For that reason, we pushed our investigation further to catch the mentioned dynamics by using a learning algorithm which, in theory, could derive a better model by learning the probability of edges connected from our data sample. Thus, if our intuition is correct, this would allow us to predict the dynamics of the Bitcoin Transactions Temporal Network. 
