\documentclass[../thesis.tex]{subfiles}

\begin{document}

\chapter{Introduction and Related Work}
\label{chap:Introduction}

``Question answering'' (QA) sites have gained considerable popularity over the last couple of 
years. In these Internet platforms users pose questions and answers to the public, in order to 
exchange knowledge. Yahoo Answers, Quora and the Stack Exchange family of sites provide platforms 
facilitating the formation of Internet communities that in turn provide natural and seamless ways 
for organizing and obtaining knowledge \cite{YahooKnowledge}. 

Dynamic aspects of such QA sites have been studied in different contexts so far. Previous works 
in this area include: studying causality aspects through quasi experimental designs \cite{Causal}; 
user churn analysis through classification algorithms such as support vector machines or random 
forests \cite{ChurnQA}; and predictions of the future value of question answer pairs, according 
to the initial activity of the question post \cite{StackLesko}. 

In contrast to previous works, where long term activity of users are being predicted, we focus 
on time series analysis related to user-defined tags. This approach offers a possibility of a 
daily detailed analysis of users behavior. We concentrate on the QA site Stackoverflow in our work, 
because this platform has an already established reputation on the web, and boasts a community of 
over 5 million distinct active users, who have provided more than 18 million answers to more than 
11 million questions. Thanks to the shear size of the corresponding data set, as well as 
the regular activity of the user base, we were able to mine temporal data in order to uncover 
defining aspects of the dynamics of the users behavior. 

Our work of analyzing the temporal dynamics of the users is divided into two 
segments. The first segment is analyzing the collective users behavior related to a 
specific tag, through modeling changes in attention as a variability in the fluctuation of time 
series of occurrences of users defined tags, which can be categorized as a special case of 
\textit{heterocedasticity} or input dependent noise. In the second segment we focus on finding 
common user posting behavior.

Due to the complexity of the user-system interaction (millions of people discuss thousands of 
topics), flexible and accurate models are required in order to guarantee a reliable analysis. 
The Bayesian setting and Gaussian Process (GP) framework \cite{GP} have shown to provide an 
accurate and flexible tool for time series analysis in the recent years, and can be used to create 
accurate models. In particular, the possibility of incorporating error ranges, as well as different 
models trough the selection of different kernels, allows interpretability of the results. 

We provide an extension of sparse input Gaussian Processes \cite{VariableSparse,Sparse} which allow 
us to model functional dependence in the time variation of the fluctuations. In practical 
experiments, we study the top 10 different tags for the Stackoverflow data set over different years, 
spanning a data set of over 2.9 million questions. We find that our model outperforms the predictions 
made by the simple GP model under variable noise. In particular, we discover weekly and seasonal 
periodicity patterns, as well as random behavior in monthly trends.

The goal of our second segment is to find common behavioral patterns among the users. This task is 
performed by clustering the time series of postings for each user, and cluster centroids are taken 
as representatives of common users behavioral patterns. Analyzing each user as time series 
of postings however is not possible, because the majority of the users (almost 94\%) on an 
individual level show a posting behavior of less then ten postings in 2014, where the posting 
behavior of all the users together show a regular activity. This behavior may occur because of many 
reasons, but mainly because of the users limited working capacity. This leads to a sparsity of data, 
that makes it difficult to analyze the users behavior with a GP model, therefore we use a point 
process. Point processes are standard models when the objects of study are the number and 
repartition of otherwise identical points on a domain, usually time or space.

Similar work has been done for analyzing user behavior \cite{malmgren2008poissonian}, where a 
Poisson Point Process (PPP) is used to explain the heavy tails in e-mail communication. However, 
the number of users that were analyzed is much smaller than we want to analyze, due to the 
computational complexity of the PPP. Recent development of a fast approximation 
\cite{samo2014scalable} of the PPP makes it possible to apply this model on much bigger scale. In 
this approximated PPP model the intensity function is modeled by our extended sparse GP, which is 
learned by using Monte Carlo sampling.

For clustering the intensity functions of every PPP we have developed a new scale invariant 
similarity measure, called Dynamic Piecewise Time similarity measure, and K-PSC clustering algorithm 
that is a generalization of the K-SC \cite{yang2011patterns}. Our algorithm is able to find more 
descriptive cluster centroids than the normal K-means clustering algorithm. Also, we do not find 
any seasonal behavior of the individual users like we have found in the collective behavior.

The theory behind the models used in this work is presented in the next chapter. We formally 
introduce the Gaussian Process framework and provide details regarding our extensions towards 
variable noise models. In section \ref{sec: poisson_process} we present brief introduction of 
Poisson Point process. Then, in section \ref{sec: clustering_time_series_user_behaviour}, we introduce 
the Dynamic Piecewise Time similarity measure and the K-PSC clustering algorithm. In chapter 
\ref{chap:coarse_grained} we show an analysis of the periodicity of the time series of tag activity 
in the Stackoverflow data set. Next, we compare our prediction results with those of other models, 
and also discuss the advantages of introducing functional dependencies on noise terms. In chapter 
\ref{chap:fine_grained} we present the common patterns of the user behavior, that we discovered 
using the Dynamic Piecewise Time similarity measure and the K-PSC clustering algorithm. Finally, we 
provide conclusions and directions for future work.

\end{document}
